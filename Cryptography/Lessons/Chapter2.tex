% === Theorem Environments ===


% =======================================================
\section{Divisions}
% =======================================================

\subsection{Divisibility}

\paragraph{Naturals and Integers.}
\[
\mathbb{N} = \{0, 1, 2, 3, \ldots\}, \quad
\mathbb{Z} = \{\ldots, -3, -2, -1, 0, 1, 2, 3, \ldots\}
\]

\paragraph{Definition.}
Let \( a, b \in \mathbb{Z} \) with \( b \neq 0 \).  
We say that **\( b \) divides \( a \)** (and write \( b \mid a \)) if there exists \( c \in \mathbb{Z} \) such that:
\[
a = bc.
\]

\paragraph{Example.}
\[
2 \mid 4 \quad \text{but} \quad 2 \nmid 5.
\]

---

\subsubsection{Even and Odd Numbers}

\paragraph{Definition.}
An integer is **even** if it is divisible by 2, otherwise it is **odd**.

\paragraph{Examples.}
\[
-4 \text{ is even}, \quad 13 \text{ is odd.}
\]
By convention, the divisors of a natural number are its positive divisors.  
\[
\text{Example: divisors of } 6 \text{ are } 1, 2, 3, 6.
\]

---

\subsection{Euclidean Division}

\begin{proposition}[Euclidean Division]
Let \( a, b \in \mathbb{N} \) with \( b > 0 \).  
Then there exist unique integers \( q \) (quotient) and \( r \) (remainder) such that:
\[
a = bq + r, \qquad 0 \leq r < b.
\]
\end{proposition}

\paragraph{Example.}
\[
a = 27, \ b = 7 \Rightarrow q = 3, \ r = 6, \quad \text{since } 27 = 7 \cdot 3 + 6.
\]

\paragraph{Observation.}
\(b\) divides \(a\) if and only if \(r = 0\).

\paragraph{Another Example.}
\[
a = 7 \times 4 - 5 = 7 \times 3 + 2,
\]
so \(q = 3\) and \(r = 2\).

---

\subsection{Euclidean Division in Practice}

\[
a = bq + r, \quad 0 \leq r < b
\quad \Longleftrightarrow \quad
\frac{a}{b} = q + \frac{r}{b}, \quad 0 \leq \frac{r}{b} < 1.
\]

Hence:
\[
q = \left\lfloor \frac{a}{b} \right\rfloor, \quad
r = a - b \left\lfloor \frac{a}{b} \right\rfloor.
\]

\paragraph{Example.}
\[
a = 15476, \quad b = 137 \Rightarrow
q = \left\lfloor \frac{15476}{137} \right\rfloor, \quad
r = 15476 - 137q.
\]

---

\subsection{Binary Representation}

\paragraph{Example.}
Every natural number \( m \) can be expressed in binary form:
\[
112 = (1110000)_2 = 1 \times 2^6 + 1 \times 2^5 + 1 \times 2^4 + 0 \times 2^3 + 0 \times 2^2 + 0 \times 2^1 + 0 \times 2^0.
\]

\paragraph{Definition.}
Let \( m \in \mathbb{N} \).  
We write:
\[
m = (m_{B-1} \ldots m_1 m_0)_2
\]
if
\[
m = m_0 + m_1 2 + \cdots + m_{B-1} 2^{B-1}, \quad m_i \in \{0, 1\}.
\]
Each \(m_i\) is called a **bit**.

---

\subsection{Binary Representation and Euclidean Division}

Binary representation arises naturally from repeated division by \(2\):
\[
a = 2q + r, \quad r \in \{0,1\}, \quad q = \left\lfloor \frac{a}{2} \right\rfloor.
\]
Applying division recursively gives:
\[
a = (r_{B-1} \ldots r_1 r_0)_2,
\]
where the \(r_i\) are remainders.

\paragraph{Example.}
\[
\begin{aligned}
112 &= 2 \times 56 + 0 \\
56  &= 2 \times 28 + 0 \\
28  &= 2 \times 14 + 0 \\
14  &= 2 \times 7  + 0 \\
7   &= 2 \times 3  + 1 \\
3   &= 2 \times 1  + 1 \\
1   &= 2 \times 0  + 1
\end{aligned}
\]
Hence \(112 = (1110000)_2.\)

---

\begin{remark}
If \(m = (m_{B-1} \ldots m_1 m_0)_2\), then \(m < 2^B\).  
Indeed:
\[
m = m_0 + m_1 2 + \cdots + m_{B-1} 2^{B-1} \le 1 + 2 + \cdots + 2^{B-1} = 2^B - 1.
\]
\end{remark}

---

\begin{proposition}[Binary Expansion]
Let \(m < 2^B\). Consider the divisions:
\[
\begin{aligned}
m &= 2q_0 + m_0, & 0 \le m_0 < 2, \\
q_0 &= 2q_1 + m_1, & 0 \le m_1 < 2, \\
q_1 &= 2q_2 + m_2, & 0 \le m_2 < 2, \\
&\ \vdots \\
q_{B-2} &= 2q_{B-1} + m_{B-1}, & 0 \le m_{B-1} < 2.
\end{aligned}
\]
Then \(q_{B-1} = 0\) and
\[
m = \sum_{i=0}^{B-1} 2^i m_i.
\]
\end{proposition}

\begin{proof}
By substitution:
\[
m = 2q_0 + m_0 = 2(2q_1 + m_1) + m_0 = \cdots = 2^{B-1} m_{B-1} + \cdots + 2m_1 + m_0.
\]
\hfill\(\square\)
\end{proof}

---

\begin{corollary}[Number of Bits]
A natural number \(m\) has a binary representation with \(B \ge 1\) bits
\[
m = (m_{B-1}\ldots m_1 m_0)_2 \quad \text{and} \quad m_{B-1} = 1
\]
if and only if
\[
2^{B-1} \le m < 2^B.
\]
In particular,
\[
B = \lfloor \log_2 m \rfloor + 1.
\]
\end{corollary}

\begin{proof}
If \(m < 2^B\), at most \(B\) bits are needed.  
If \(B-1\) bits were enough, then \(m < 2^{B-1}\), contradiction.  
\hfill\(\square\)
\end{proof}

---

\subsection{Example}
\[
\log_2(368{,}932) = 18.4 \Rightarrow B = \lfloor 18.4 \rfloor + 1 = 19.
\]
Hence \(368{,}932\) requires 19 bits:
\[
368{,}932 = (1011000001000100100)_2.
\]

% =======================================================
\section{Greatest Common Divisor (GCD)}
% =======================================================

\subsection{Definition}

\paragraph{Example.}
The divisors of \(12\) are \(1, 2, 3, 4, 6, 12\).  
The divisors of \(18\) are \(1, 2, 3, 6, 9\).  
Their common divisors are \(\{1,2,3,6\}\), whose largest is 6:
\[
6 := \gcd(18, 12).
\]

\begin{remark}
When we write \(\gcd(a,b)\), we assume \(a\) or \(b\) is nonzero.
\end{remark}

\paragraph{Definition.}
Let \(a,b \in \mathbb{Z}\), not both zero.  
The set of common divisors of \(a,b\) has a largest element \(d\), called the **greatest common divisor** of \(a,b\):
\[
d = \gcd(a,b).
\]

\paragraph{Examples.}
\[
\gcd(12,0)=12, \qquad \gcd(0,12)=12.
\]

---

\subsection{The Euclidean Algorithm}

\paragraph{Example.}
Find \(\gcd(119,259)\):
\[
259 = 2\cdot119 + 21, \quad 119 = 5\cdot21 + 14, \quad 21 = 1\cdot14 + 7, \quad 14 = 2\cdot7 + 0.
\]
The last nonzero remainder is 7:
\[
\boxed{\gcd(119,259)=7.}
\]

\begin{theorem}[Euclidean Algorithm]
Let \(a,b>0\) with \(a\ge b\).  
Then \(\gcd(a,b)\) is found by:
\[
r_0=a, \ r_1=b, \quad r_{i-1}=r_i q_i + r_{i+1}, \ 0\le r_{i+1}<r_i.
\]
If \(r_{i+1}=0\), then \(r_i=\gcd(a,b)\).
\end{theorem}

\begin{proof}
The sequence \(r_0>r_1>\cdots\ge0\) terminates.  
Let \(d=r_{n+1}\) be the last nonzero remainder.  
Then \(d\mid r_i\) for all \(i\), so \(d\mid a,b\).  
If \(c\mid a,b\), then \(c\mid d\).  
Hence \(d=\gcd(a,b)\).  
\hfill\(\square\)
\end{proof}

---

\section{The Extended Euclidean Algorithm}

\paragraph{Example.}
Compute \(\gcd(a,b)\) as a linear combination:
\[
a=259,\ b=119
\]
\[
\begin{aligned}
259 &= 2\cdot119 + 21,\\
119 &= 5\cdot21 + 14,\\
21  &= 1\cdot14 + 7,\\
14  &= 2\cdot7 + 0.
\end{aligned}
\]
Expressing each remainder:
\[
\begin{aligned}
21 &= a - 2b,\\
14 &= b - 5(a-2b) = -5a + 11b,\\
7  &= (a-2b) - (-5a+11b) = 6a - 13b.
\end{aligned}
\]
Thus:
\[
\boxed{\gcd(a,b)=7=6a-13b.}
\]

\begin{proposition}[Extended Euclidean Algorithm]
If \(d=\gcd(a,b)\), then there exist \(u,v\in\mathbb{Z}\) such that
\[
\boxed{d=au+bv.}
\]
\end{proposition}

\begin{proof}
From the recursive divisions:
\[
r_{i-1}=r_i q_i + r_{i+1},
\]
each remainder is a linear combination of \(a,b\).  
By recursion, the final nonzero remainder \(r_n=d\) satisfies
\[
d = a u + b v.
\]
\hfill\(\square\)
\end{proof}
