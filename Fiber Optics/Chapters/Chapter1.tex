%==============================
% Chapter 1 - Basics of EM Waves
%==============================

\chapter{Basics of Electromagnetic Waves}

\section*{Maxwell’s Equations (Time Domain)}

Electromagnetic phenomena are described by **Maxwell’s equations**, which relate the electric and magnetic fields and their variations in space and time:

\begin{align}
\nabla \times \vec{e} &= - \frac{\partial \vec{b}}{\partial t} \\
\nabla \times \vec{h} &= \vec{j} + \frac{\partial \vec{d}}{\partial t} \\
\nabla \cdot \vec{d} &= \rho \\
\nabla \cdot \vec{b} &= 0
\end{align}

where:
\begin{itemize}
    \item $\vec{e}(\vec{r},t)$ is the electric field [V/m],
    \item $\vec{d}(\vec{r},t)$ is the electric flux density [C/m$^2$],
    \item $\vec{h}(\vec{r},t)$ is the magnetic field [A/m],
    \item $\vec{b}(\vec{r},t)$ is the magnetic flux density [Wb/m$^2$],
    \item $\rho(\vec{r},t)$ is the charge density [C/m$^3$],
    \item $\vec{j}(\vec{r},t)$ is the electric current density [A/m$^2$],
    \item $\vec{r} = (x, y, z)$ is the position vector.
\end{itemize}

\noindent
The differential operator $\nabla$ (nabla) is defined as:
\[
\nabla = 
\begin{bmatrix}
\frac{\partial}{\partial x} \\
\frac{\partial}{\partial y} \\
\frac{\partial}{\partial z}
\end{bmatrix}
\]
and acts as:
\[
\nabla \cdot \vec{a} = 
\frac{\partial a_x}{\partial x} +
\frac{\partial a_y}{\partial y} +
\frac{\partial a_z}{\partial z}, \qquad
\nabla \times \vec{a} = 
\begin{bmatrix}
\frac{\partial a_z}{\partial y} - \frac{\partial a_y}{\partial z} \\
\frac{\partial a_x}{\partial z} - \frac{\partial a_z}{\partial x} \\
\frac{\partial a_y}{\partial x} - \frac{\partial a_x}{\partial y}
\end{bmatrix}
\]

Two useful vector identities to remember:
\[
\nabla \cdot (\nabla \times \vec{A}) = 0,
\qquad
\vec{A} \cdot (\nabla \times \vec{B}) = 0
\]

---

\section*{Time-Harmonic Fields}

For time-harmonic fields, we can express a time-varying electric field as:
\[
\vec{e}(t) = \Re \left[ \vec{E} e^{j\omega t} \right]
\]
where $\omega = 2\pi f$ is the **angular frequency**.

Expanding:
\[
\vec{E} = \vec{E}_R + j\vec{E}_I, \qquad
e^{j\omega t} = \cos(\omega t) + j\sin(\omega t)
\]
hence:
\[
\vec{e}(t) = \vec{E}_R \cos(\omega t) - \vec{E}_I \sin(\omega t)
\]

Taking the time derivative:
\[
\frac{\partial \vec{e}}{\partial t} = 
\Re \left[ j\omega \vec{E} e^{j\omega t} \right]
\]

Substituting into Maxwell’s first equation:
\[
\nabla \times \vec{e} = -\frac{\partial \vec{b}}{\partial t}
\]
and using the harmonic representation:
\[
\nabla \times \vec{E} = -j\omega \vec{B}
\]

---

\section*{Maxwell’s Equations (Harmonic Domain)}

In many practical cases, electromagnetic fields can be represented by **sinusoidal** (harmonic) signals.  
Using **complex Steinmetz vectors**, fields are expressed as:
\[
\vec{e}(\vec{r}, t) = \Re [ \vec{E}(\vec{r}) e^{j\omega t} ]
\]

Assuming no electric charges or currents ($\rho = 0$, $\vec{J} = 0$), Maxwell’s equations simplify to:
\begin{align}
\nabla \times \vec{E} &= -j\omega \vec{B} &
\nabla \cdot \vec{D} &= 0 \\
\nabla \times \vec{H} &= j\omega \vec{D} &
\nabla \cdot \vec{B} &= 0
\end{align}

---

\section*{Constitutive Relations}

In dielectric (non-conductive) media, there are **no magnetic fields generated by conduction currents**.  
The relations between field vectors are:

\[
\vec{B} = \mu_0 \vec{H}, \qquad
\vec{D} = \varepsilon \vec{E}
\]

where $\varepsilon$ is the **dielectric permittivity** and $\mu_0$ is the **magnetic permeability of free space**.

For **isotropic** materials:
\[
\vec{D} = \varepsilon \vec{E}
\]

For **dispersive** materials (frequency-dependent properties):
\[
\vec{D}(\omega, \vec{r}) = \varepsilon(\omega, \vec{r}) \vec{E}(\omega, \vec{r})
\]
with position vector
\[
\vec{r} = 
\begin{bmatrix}
x \\ y \\ z
\end{bmatrix}
\]

---

\noindent\textit{[End of Chapter 1 excerpt]}
